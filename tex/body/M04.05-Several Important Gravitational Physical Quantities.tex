% 136.jpg
\section{几个重要的引力物理量}\label{sec:04.05}

现在我们再回到引力问题。任何一种相互作用,都决定着一
类运动过程。例如,行星的运动、月亮的运动以及落体运动都取
决于万有引力。这些运动的特征尺度,主要决定于引力常数,所
以它们也可以称为引力物理中的特征量。本节就来讨论一些常见
的特征尺度。

所谓第一宇宙速度,就是一个引力物理量。如图\ref{fig:04.03},我们已
经讨论过当抛体速度达到一定值时,它就会绕地球作匀速圆周运
动而不再落回地面,成为地球的卫星。这个速度称为第一宇宙速
度。

按第一宇宙速度$ v $的定义,当物体达到速率$ v $时,它将绕地球
作匀速圆周运动。这时,地球和物体之间的距离可用地球半径$ R $
代替,根据牛顿第二定律和万有引力定律,得
\begin{equation*}
  \frac { G M _ \text{地} m } { R ^ { 2 } } = m \frac { v ^ { 2 } } { R } = m g
\end{equation*}
其中$ M _ \text{地} $是地球的质量,$ m $是物体的质量,所以
\begin{equation}\label{eqn:04.05.01}
  v = \sqrt {  \frac { G M _ { \text{地} } } { R }}
  \quad \text{或} \quad v = \sqrt { R g }
\end{equation}
用$ R\approx 6400 $公里, $ g \approx 9.8 $米/秒$ ^2 $,代入上式得到
\begin{equation*}
  v \approx 7.9 \;\text{公里/秒}
\end{equation*}
这就是第一宇宙速度的数值。

要使物体能逃离地球,不再返回,它的速度应更高,即至少
要具有第二宇宙速度那样高的值才行。所谓第二宇宙速度,由下
式给出,
\begin{equation*}
  v = \sqrt{\frac { 2 G M _ { \text{地} } } { R }} \approx 11 \;\text{公里/秒}
\end{equation*}
% 137.jpg
有关这个公式的证明,留给读者。

对于一个质量为M,半径为$ r $的体系,同样可以规定它的
“第一宇宙速度”为
\begin{equation}\label{eqn:04.05.02}
  v _ { 1 } = \sqrt {\frac { G M } { r }}
\end{equation}
以及“第二宇宙速度”为
\begin{equation}\label{eqn:04.05.03}
  v _ { 2 } = \sqrt {\frac { 2 G M } { r }}
\end{equation}
$ v_1 $和$  v _ 2 $的物理意义与前面讨论的相似。当物体的速度达到$ v_1 $时,
即可绕$ M $运动而不会落到$ M $上;当物体速度大到$ v_2 $时,它即可逃
离$ M $,而不再被吸引回来。

下面我们利用$  v _ 1 , v _ 2 $来讨论一些有趣的现象。

如果有一个引力体系的第二宇宙速度等于光速,有
\begin{equation*}
  c = \sqrt { \frac { 2 G M } { r } } \quad
  \text{或} \quad
  r = \frac { 2 G M } { c ^ { 2 } }
\end{equation*}
这样,一个体系若满足下列不等式
\begin{equation*}
  r < \frac { 2 G M } { c ^ { 2 } }
\end{equation*}
它的第二宇宙速度$ v_2 $就要大于光速。这就是说,在这种物体上发
射的光也都不能克服引力的作用,最终一定要落回到该体系上来。
简言之,这种物体根本不可能有光发射出去,因此,我们不能看
到它,故称为黑洞。因此,
\begin{equation}\label{eqn:04.05.04}
  r _ { g } = \frac { 2 G M } { c ^ { 2 } }
\end{equation}
也是一个关键性的物理量,称为引力半径。对于地球,质量$ M _ { \text{地} }= 6 \times 1 0 ^ { 2 7 }  $克,代入式\eqref{eqn:04.05.04},可求得地球的引力半径是$  r _ { g } \approx 0 . 9  $
厘米。

上述计算表明,如果地球的全部质量能缩小到半径约$ 1 $厘米
% 138.jpg
的小球内,那么,生活在这样小球上的人,将无法和外界进行光

\clearpage\noindent
的或无线电的联系,它将成为一个孤立的体系。这当然只是一个
设想。下面我们来看一个较实际的例子。

我们考虑一个球体,其半径为$ r $,其中物质均匀分布,密度
为$ \rho $,则体系的质量为
\begin{equation}\label{eqn:04.05.05}
  M = \frac { 4 \uppi } { 3 } r ^ { 3 } \rho
\end{equation}
如果这个体系的半径恰好达到自己的引力半径,则由式\eqref{eqn:04.05.04} 、\eqref{eqn:04.05.05}得到
\begin{equation*}
  r _ { g } = \frac { 8 \uppi G r _ { g } ^ { 3 } \rho } { 3 c ^ { 2 } }
\end{equation*}
亦即引力半径为
\begin{equation}\label{eqn:04.05.06}
  r _ { g } = \left( \frac { 3 c ^ { 2 } } { 8 \uppi G \rho } \right) ^ { \frac { 1 } { 2 } }
\end{equation}
式\eqref{eqn:04.05.06}是说,对于生活在密度为$ \rho $的环境中的人,他不可能把
光发射到超出$ r_g $的范围。我们生活的宇宙环境的密度平均约为$ \rho
  \approx 10 ^ {-29}$克/厘米,因此,由式\eqref{eqn:04.05.06}立即求得引力半径为
\begin{equation*}
  r _ g \approx 10 ^ {28}\;\text{厘米}
\end{equation*}
这就是说,我们不可能把光发射到$ 10 ^ {28} $厘米之外,我们称这个尺
度为宇宙大小,或宇宙半径。