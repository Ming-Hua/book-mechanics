\achapter

\addtocounter{answer}{4}

\answer $ I = \dfrac { 1 } { 1 2 } m \Bigl( a ^ 2 + b ^ { 2 } - \dfrac { 2 c ^ { 4 } } { a b } \Bigr) $

\answer $ I = \dfrac { m } { 1 2 a b } \left( a ^ { 3 } b + b ^ { 3 } a - 6 \uppi r ^ { 4 } \right) $
% 396.jpg

\answer (1) $ 1 6 \times 1 0 ^ { 4 } $克$ \cdot $厘米\textsuperscript{2}

(2) $ 1 . 2 \times 1 0 ^ { 4 } $克$ \cdot $厘米\textsuperscript{2}

\answer $ I = \dfrac { 1 } { 2 } - m \left( R _ { 1 } ^ { 2 } + R _ { 2 } ^ 2 \right) $

\answer $ I = \dfrac { 1 } { 4 } m R ^ { 2 } $

\answer $ I = \dfrac { m } { 12 } h ^ { 2 } + \dfrac { 1 } { 4 } m R ^ { 2 } $

\answer $ \beta = 3 . 2 8 $弧度/秒\textsubscript{2}

\answer $ a _ { c } = \dfrac { 5 } { 7 } g \sin \alpha $

\answer $ \beta = - 2 . 1 $弧度/秒\textsubscript{2}, $ n = 7 0 . 8 $转,再经$ 40 $秒停止

\answer $ v = 2 \sqrt { \dfrac { m g h } { 2m + M } } $

\answer $T=\dfrac{m_{1} \mu+m_{1}+M \mu}{m_{1}+m_{2}+M} g, \quad T^{\prime}=\dfrac{m_{2}+m_{2} \mu+M}{m_{1}+m_{2}+M} g$

$a=\dfrac{m_{1}-m_{2} \mu}{m_{1}+m_{2}+M} g$

\answer $ a = - \dfrac { 2 } { 3 } g $ , $ T = \dfrac { 1 } { 3 } m g $

\answer $ f = 2 9 $公斤力

\answer $v=\sqrt{\dfrac{4}{3} g h}=2 \sqrt{\dfrac{g h}{3}}$

\answer $ f = 3 . 5 9 \times 1 0 ^ { 3 } $牛顿

\answer $a=\dfrac{4(2 M-m)}{8 M+7 m} g, \quad T_{1}=\dfrac{m g}{8 M+7 m}(5M+3 m)$

$T_{2}=\dfrac{m g}{8 M+7 m}(7 M+2 m), \quad T_{3}=\dfrac{11 M m}{8 M I+7 m} g$

\answer (1) $ a = 2 . 3 3 $米/秒\textsubscript{2}

(2) $ T_ { 1 } = 3 5 $牛顿, $ T _ { 2 } = 3 7 . 3 2 $牛顿

\answer 转了 $ \dfrac { 1 } { 2 \uppi } \sqrt { \dfrac { 3 h } { a } } $转
% 397.jpg
\clearpage
\answer $ M = - 0 . 0 8 $牛顿$ \cdot $米,其中负号表示与原转动角动量反方向

\answer (1) $ v _ { B } = 1 2 . 1 $米/秒

(2) $ y _ { H } = 3 . 3 3$米

\answer 球先到下端,$ v _ { c \text{质点} } > v _ { c \text{球} } > v _ { c \text{柱} }$

\answer (1) $a_{c 1}=\dfrac{5}{7} g \sin \alpha, \quad v_{c 1}=\sqrt{\dfrac{10}{7} g h}, \quad \omega_{1}=20 \sqrt{\dfrac{10}{7} g h}$

\aindent $a_{c 2}=\dfrac{5}{7} g \sin \alpha, \quad v_{c 2}=\sqrt{\dfrac{10}{7} g h}, \quad \omega_{2}=10 \sqrt{\dfrac{10}{7} g h}$

\aindent $a_{c 3}=\dfrac{3}{5} g \sin \alpha, \quad v_{c 3}=\sqrt{\dfrac{6}{5} g h}, \quad \omega_{3}=20 \sqrt{\dfrac{6}{5} g h}$

\aindent $a_{c 4}=\dfrac{3}{5} g \sin \alpha, \quad v_{c 4}=\sqrt{\dfrac{6}{5} g h}, \quad \omega_{4}=10 \sqrt{\dfrac{6}{5} g h}$

(2) 同样结构的球(实或空)$ a _ c $与$ m,r $无关,
但$ \omega \propto \dfrac { 1 } { r } $\vspace{-0.5em}

\aindent 同样的材料、同样尺寸,则$
  a _ { c }, \omega \propto \dfrac { 1 } { I } $

\answer $ t = 2 . 1 $秒

\answer {\ziju{-0.02pt}$ \omega _ { 0 } > \dfrac { 5 } { 2 } \cdot \dfrac { v _ { 0 } } { R } $时,\!成为“来去”;$ \omega _ { 0 } = \dfrac { 2 } { 5 } \cdot \dfrac { v _ { 0 } } { R } $,\!则同时停转又停止}

\answer $ \mu \geqslant \dfrac { 1 } { 3 } \tg \alpha $\vspace{-1em}
