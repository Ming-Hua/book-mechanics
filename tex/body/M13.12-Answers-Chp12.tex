\achapter

\answer (1)7.66公斤(2)5.31公斤力

\answer 4.64×102米/秒

\answer (1) $ \alpha = 0 , T = m g $

(2) $ \alpha = 0 , T = m g $

(3) $ \alpha = \tg ^ { - 1 } \dfrac { a } { g }, T = m \sqrt { a ^ { 2 } + g ^ { 2 } } $

(4) $ \alpha = \theta , T = m g \cos \theta $

(5) $ \alpha = \tg ^ { - 1 } \dfrac { b \cos \theta } { g + b \sin \theta } ,
  T = m \sqrt { g ^ { 2 } + b ^ { 2 } + 2 b g \sin \theta } $

(6) $ \alpha = \tg ^ { - 1 } \dfrac { b \cos \theta } { g - b \sin \theta } ,
  T = m \sqrt { g ^ { 2 } + b ^ { 2 } - 2 b g \sin \theta } $ \\
式中$ \alpha $为小球处于平衡位置时悬线与指向地面铅垂线之间的夹角

\answer (1) $ U = m g J \left( 1 - \cos \alpha \right) $,设摆锤在最低点时位能为零

(2) $\displaystyle A = \int_{ 0 } ^ \infty m a l \cos \alpha \dif \alpha = m a l sin \alpha $

(3) $ a _ { \text{max} } = 2 \tg ^ { - 1 } \dfrac { a } { g }$

(4) $\alpha _ { 0 } = \tg ^ { - 1 } \dfrac { a } { g } $,所以$ \alpha _ { \text{max} } = 2 \alpha _ { 0 } $
% 399.jpg
放手后,摆锤以$ \alpha_{ 0 } $位置为平衡点,作振幅为$ \dfrac { 1 } { 2 } \alpha _ { \text{max} } $的周期振动

\answer 见第三章例7

\answer 设$ g_0 $为地球海平面的重力加速度,$ g $为飞机所在处的重力加速度,
$ \vec{F} $为驾驶员作用于座位上的力,$\vec{F}$与向下的竖直方向夹角为$\varphi$

(1) $F=\dfrac{R}{g_{0}} \sqrt{a^{2}+g^{2}+2 a g \cos \theta}, ~ \varphi=\operatorname{tg}^{-1} \dfrac{a \sin \theta}{g+a \cos \theta}$

(2) $ F = P \dfrac { g + a } { g _ { 0 } } , ~ \varphi = 0 $

(3) $ F = \dfrac { P } { g _ { 0 } } \sqrt { a ^ { 2 } + g _ { 2 } }, ~ \varphi = t g ^ { - 1} \dfrac { a } { g } $

(4) $ F = 0 $

(5) $ F = P \dfrac { g } { g _ { 0 } } , \varphi = 0 $

\answer (1) $ P _ { \text{min} } = 1 . 9 7 $ 公里

(2) $ 300 $公斤

\stepcounter{answer}
\answer (1) $ y = \sqrt { \dfrac { g R } { \mu _ { s } } } $

(2) $ v = 14 $ 米/秒

(3) $ a = 1 1 . 0 9 ^ { \circ } \approx 1 1 ^ { \circ } $

\stepcounter{answer}
\answer 水平方向向东 $ \Delta s = 2 2 . 7 $米

\answer (1) $\wideparen{A B} = \omega \dfrac { R ^ { 2 } } { v }$

(2) $ \wideparen{A B} = 4 \omega \dfrac { R ^ { 2 } } { v } $

\answer $\vec{F} = \bigl( \underbrace{- 0 . 0 5 \vec{j}}_{\text{向南}} + \hspace{-11.6ex} \underbrace{0 . 0 6 \vec{k}\vphantom{\vec{j}}} _{\text{\hspace{7em}向上,抵消一部分重力}}\hspace{-11.4ex}\bigr)$牛顿

\answer $ a = g \theta $, $ \theta = 1 . 5 6 \times 1 0 ^ { - 1 1 } $弧度, 所以$ a = 1 . 5 3 \times 1 0 ^ { - 1 0 } $米/秒\textsuperscript{2},方向沿
着地面二者之联线
