\setcounter{page}{1}
\pagestyle{foreword}
\null\vspace{1em}
\begin{center}
  \label{foreword}\pdfbookmark{序}{foreword}
  \zihao{4}\heiti{序}

  \null{}
\end{center}
\fangsong\normalsize

这本书原是一份普通物理课程的力学讲义,它曾在中国科学
技术大学沿用多年,也曾在北京大学教授过数次。

普通物理中的力学,是相当难教的,凡是教授过这门课的老
师,大都有此体会。一方面,力学是整个物理学的基石,它包含
许多基本的观念、方法和理论,需要学生极为准确地加以掌握,
以备后继学习之用,另一方面,初入大学的学生,往往看轻力学,
误认为新的内容不多,似乎在中学里都已学过,结果力学反而被
疏忽了。

这种局面迫使一些教师采用理论力学的方法来教授普通物理
力学。这样做,确实可以解决前述问题的第二方面,学生不再感
到“似曾相识”了。随着教和学二者的提高,原属理论力学的部
分内容的确可以逐渐放到普通物理中来。但是,我们觉得,若仅
限于这一途径改进教学,还不能或不完全能解决问题的第一个方
面——力学是整个物理的一块基石。

基石到底在哪里起了基石的作用?基石到底如何起了基石的
作用?显然,这些“哪里”,这些“如何”只有从物理的当代发
展以及前沿研究的角度,才能看得清楚。这就是说,如果我们企图
从“物理的基石”这一标准来组织教学,它至少有以下两方面的
含义:一是不断用新的现代的观点去整理老的内容,一是不断用新
的前沿的重要成果来充实基础。事实上,不同时代的教材的差别,
最清楚地表现在这些方面。上进的标准,也就是我们在编写这本
教材时,尝试着击追求的。也许有的地方达到了,也许有的地方%分页处
并未达到。无论成功或失败,它都是我们的追求的记录。

为了使用上的方便,书中编辑了一些例题,每章末也附有一
些思考题和习题。由于北京大学物理系和中国科学技术大学物理
教研室已编有《物理学习题集》(人民教育出版社,1980),为了不
重复太多,本书中的例题和习题只是标志性的。在教学上需要更
多习题时,可以参考上述的习题集。

在使讲义变成这本书的过程中,得到过员汝槐同志的协助,
谨致谢意。

\null{}

\hspace{6.8cm}\zihao{-4}\kaishu{作~~~者}

\mbox{}

\hspace{7cm}\normalfont{} \zihao{-5}1984年4月\normalsize
\clearpage
