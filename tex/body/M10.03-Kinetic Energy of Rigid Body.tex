\section{刚体的动能}\label{sec:10.03}

刚体是由许多质点组成的。所以,刚体的动能就等于各质点
的动能之和,即
\begin{equation}\label{eqn:10.03.01}
  T = \sum _ { i = 1 } ^ { n } \dfrac { 1 } { 2 } m _ i v _ i ^ { 2 }
\end{equation}

上节中已经指出,刚体的运动总可以分解为随着某基点$ O $的
平动和绕该点的转动,而且基点是可以任意选取的。自然会问:
动能\lhbrak 式\eqref{eqn:10.03.01}\rhbrak 是否也可分解成平动动能和转动动能呢?这是
可以做到的。但对基点的选取有特殊的要求,即只有当选取刚体
的质心$ C $作为基点时,才可以进行这种分解。下面我们来证明这
一点。

刚体中每个质点的速度可分解成平动部分及转动部分\lhbrak 式 \eqref{eqn:10.02.06}\rhbrak 。如果我们取质心$ C $作为基点,那么,平动部分的速度
就是质心速度$ \vec{v} _ c $,转动部分的速度就是质点相对于质心系的运动
速度$ \vec{v}' _i $。根据速度合成,有
\begin{equation*}
  \vec{v} _ { i } = \vec{v} _ { c } + \vec{v} _ { i } ^ { \prime }
\end{equation*}
所以,式\eqref{eqn:10.03.01}可以改写为
\begin{equation}\label{eqn:10.03.02}
  \begin{split}
    T &=\sum_{i=1}^{n} \dfrac{1}{2} m_{i}\left(\vec{v}_{c}+\vec{v}_{i}^{\prime}\right)^{2} \\
    &=\sum_{i=1}^{n} \dfrac{1}{2} m_{i}\left(\vec{v}_{c}^{2}+\vec{v}_{i}^{2 \prime}+2 \vec{v}_{c} \cdot \vec{v}_{i}^{\prime}\right) \\
    &=\sum_{i=1}^{n} \dfrac{1}{2} m_{i} v_{c}^{2}+\sum_{i=1}^{n} \dfrac{1}{2} m_{i} v_{i}^{\prime 2}+\sum_{i=1}^{n} m_{i} \vec{v}_{i}^{\prime} \cdot \vec{v}_{c}
  \end{split}
\end{equation}
因为
$ \vec{v} _ i ^ { \prime } = \dfrac { \dif \vec{r}'_ i } { \dif t } $,
$ \vec{r}'_i $是第$ i $个质点相对于质心的位置矢量,所以
% 294.jpg

~\vspace{-1.5em}
\begin{equation*}
  \sum _ { i = 1 } ^ { n } m _ { i } \vec{v}' _ { i } = \dfrac { \dif } { \dif t } \sum_{ i = 1 }^ n m _ i \vec{r}_i
\end{equation*}
按质心的定义,应有
\begin{equation*}
  \sum _ { i = 1 } ^ { n } m _ { i } \vec{r} _ { i } ^ { \prime } = 0
\end{equation*}
所以,式\eqref{eqn:10.03.02}中最后一项总为零,故有
\begin{equation}\label{eqn:10.03.03}
  \begin{split}
    T &= \dfrac { 1 } { 2 } \Big( \sum _ { i = 1 } ^ { n } m _ i \Big) v _ { c } ^ { 2 } + \dfrac { 1 } { 2 } \sum_{ i = 1 } ^ n m _ i v _ i ^ { \prime 2 } \\
    &= \dfrac { 1 } { 2 } m v _ c ^ { 2 } + \dfrac { 1 } { 2 } \sum_{ i = 1 } ^ n m _ i v _ i ^ { \prime 2 }
  \end{split}
\end{equation}
其中$ m = \sum\limits_{ i = 1 } ^ n m _ i $是刚体的总质量。上式表示刚体的动能的确可以看
成两部分的和,即以质心为基点的平动动能$ \dfrac { 1 } { 2 } m v _ c ^ { 2 } $及相对于质心
的转动能
\begin{equation}\label{eqn:10.03.04}
  T _ \text{转} = \frac { 1 } { 2 } \sum_{ i = 1 } ^ n m _ i v _ i ^ { \prime 2 }
\end{equation}
如果转动角速度为$ \omega $,则
\begin{equation}\label{eqn:10.03.05}
  v _ i ^ { \prime } = r _ i \omega
\end{equation}
其中$ r _ i $是第$ i $个质点与转动轴的距离。将式\eqref{eqn:10.03.05}代入式\eqref{eqn:10.03.04},得到
\begin{equation}\label{eqn:10.03.06}
  \begin{split}
    T _ \text{转} &= \dfrac { 1 } { 2 } \Big( \sum_{ i = 1 } ^ n m _ i r _ i ^ 2 \Big) \omega ^ { 2 } \\
    &= \dfrac { 1 } { 2 } I _ { c } \omega ^ { 2 }
  \end{split}
\end{equation}
\begin{align}\label{eqn:10.03.07}
  \beforetext{其中} I _ { c } = \sum_{ i = 1 } ^ n m _ i r _ i ^ 2
\end{align}
按式\eqref{eqn:09.04.09},它就是该刚体相对于转动为$ \omega $的轴的转动惯量。由
% 295.jpg
式\eqref{eqn:10.03.04}、 \eqref{eqn:10.03.06} ,刚体动能\lhbrak 式\eqref{eqn:10.03.03}\rhbrak 可以写成
\begin{equation}\label{eqn:10.03.08}
  T = \frac { 1 } { 2 } m v _ { c } ^ { 2 } + \frac { 1 } { 2 } I _ { c } \omega ^ { 2 }
\end{equation}
这就是刚体动能的最终表达式。应当再强调一遍,只有选择质心
作为分解平动及转动的基点时,式\eqref{eqn:10.03.08}才是适用的。

现在,我们从动能角度再来讨论一下上节中的车轮纯滚动例
子。我们已经强调了取质心作为基点对于描写刚体的动能有很大
的优越性。这时刚体动能由\eqref{eqn:10.03.08}表示。但原则上说,为描写
刚体动能,基点总是可以任意选取的。如果我们取车轮与地面的
接触点$ A $为基点,则各质点相对地面的速度$ \vec{v} _ i $可写为
\begin{equation*}
  \vec{v} _ i = \vec{v} _ { A } + \vec{v}' _ i
\end{equation*}
式中$ \vec{v}' _ i $是质点$ m _ i $相对于基点$ A $的速度;$ \vec{v} _ A $是基点$ A $相对于地面的速
度。上节已证明车轮与地面接触处相对地面的速度为零,即$ \vec{v} _ { A } =
  0$, 所以
\begin{equation*}
  \begin{split}
    T &=\sum_{i=1}^{n} \dfrac{1}{2} m_{i} \vec{v}_{i}^{2} \\
    &=\sum_{i=1}^{n} \dfrac{1}{2} m_{i} \vec{v}_{i}^{\prime 2} \\
    &=\sum_{i=1}^{n} \dfrac{1}{2} m_{i} v_{i}^{\prime 2} \\
    &=\sum_{i=1}^{n} \dfrac{1}{2} m_{i} r^{2}_{Ai} \omega_{A}^{2} \\
    &=\dfrac{1}{2}\Big(\sum_{i=1}^{n} m_{i} r_{Ai}^{2}\Big) \omega_{A}^{2} \\
    &=\dfrac{1}{2} I^{\prime} \omega_{A}^{2}
  \end{split}
\end{equation*}
其中$ r_{Ai} $是质点$ m _ i $到通过$ A $的转轴的距离;$ I' $是车轮对通过$ A $且垂直
于轮面的轴的转动惯量;$ \omega_{A} $是以$ A $为基点的角速度。我们已经知
% 296.jpg
道,刚体的角速度与基点的选取无关。故$ \omega _ { A } = \omega $, 所以\vspace{-1.56em}
\begin{equation}\label{eqn:10.03.09}
  T = \frac { 1 } { 2 } I ^ { \prime } \omega ^ { 2 }
\end{equation}
比较式\eqref{eqn:10.03.08}与式\eqref{eqn:10.03.09},得
\begin{equation*}
  \frac { 1 } { 2 } I ^ { \prime } \omega ^ { 2 } = \frac { 1 } { 2 } m v _ { c } ^ { 2 } + \frac { 1 } { 2 } I _ { c } \omega ^ { 2 }
\end{equation*}
\begin{align*}
  \beforetext{又} v _ { c } = r _ { 0 } \omega
\end{align*}
\begin{align}\label{eqn:10.03.10}
  \beforetext{故} I ^ { \prime } = m r _ 0 ^ { 2 } + I _ { c }
\end{align}
式\eqref{eqn:10.03.10}表明,\!车轮绕过$ A $点的轴的转动惯量$ I' $与绕过质心
的、\!与之平行的轴的转动惯量$ I _ c $之间有一个简单关系式。\!可以证
明,\!上述关系是普遍成立的,\!即对于任何刚体,\!绕任意转轴的转
动惯量,\!式\eqref{eqn:10.03.10}都成立,其中$ r _ 0 $是质心与该转轴之间的距离。

