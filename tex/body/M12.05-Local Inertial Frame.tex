\section{局部惯性系}\label{sec:12.05}

等效原理保证了在任何一个局部范围中,一定存在着引力作
用都被消除的参考系。在这种参考系中,一切不受外力作用的质
点,都作匀速直线运动。所以,按照惯性系的定义,这种参考系
应是一个惯性系。它被称为局部惯性系。

% 371.jpg
\clearpage
在牛顿力学中,用惯性定律来判断一个参考系是不是惯性
系,即在没有外力的环境中,质点应保持惯性运动。但是,由于
引力是不可屏蔽的,它无处不在。所谓“没有外力的环境”实质
上是一个不存在的环境。因此,将惯性系建立在这种条件上,原
则上是缺乏根据的(尽管可以选择一些“实用”惯性系,近似满足
这个条件)。但在局部惯性系中,我们才真正能找到“没有外力的
环境”,并且在这个环境中的确仍有惯性定律。因此,局部惯性
系更加接近惯性系的本来要求。

局部惯性系比牛顿体系中的惯性系概念更明确也更本质。首
先,局部惯性系概念说明,由于引力的存在,只有在局部范围中
才能使用惯性系的概念,牛顿体系中所假定的大范围的、甚至全
空间统一的惯性系,在原则上是不存在的。

其次,在牛顿体系中我们不清楚为什么惯性系特别“优越”
和“独特”,牛顿用绝对空间来解释这一点,而绝对空间本身却
是更不清楚的。现在我们看到,局部惯性系之所以特别,因为在
这种参考系中引力没有了。所以,对物体运动的描写大大简化。

第三,在牛顿体系中,惯性系是决定于绝对空间的,但它本
身却不受物质运动的影响。亦即绝对空间是一个物理实在,因为
它会影响物体的动力学性质,决定动力学方程的形式,这是十分
强的影响。但是,物质运动却不能对绝对空间有任何影响。这种
没有反作用的单向关系,与一般物理规律的特征相当不协调。在
局部惯性系体系中,一个作自由落体运动的实验室才是一个局部
惯性系,显然,它是决定于物质的分布及运动的。现在我们既不
要求局部惯性系相对于某个绝对空间是无加速度的,也不要求各
个不同的局部范围上的惯性系之间是无加速度的。例如,围绕地
球运行的人造卫星、飞向金星的飞船,它们都是局部惯性系,因
为它们都是在纯引力的作用下作自由的飞行,尽管它们之间可能
是有加速度的。

% 372.jpg
\clearpage
总之,引力的作用使大范围的惯性系不再存在,只能有局部
的惯性系。引力的作用就在于决定各个局部惯性系之间的联系。
在任何一个局部惯性系中,我们是看不到引力作用的,只能在各
个局部惯性系的相互关系中才能看到引力的作用。

总结经典力学的发展。在牛顿体系中,工作程序总是这样的:
取定一定的参考系用以度量有关的物理量,然后给出力的性质,
写出动力学的基本方程。在这个过程中,时空的几何性质(即由
所取的参考系决定)是不受有关的物理过程影响的。

但是,爱因斯坦的理论表明,引力一方面要影响物体的运动,
另一方面又要影响各局部惯性系之间的关系。所以,我们不可能
先行规定时空的几何性质,或先行规定参考系。这种先行规定的
东西可能并不存在,时空的几何性质本身就是有待确定的东西。
这种新的力学,不仅讨论物体之间的相互作用,而且讨论物质运
动与时空几何之间的关系,时空本身也成了一种动力学的量。这
种力学,就是爱因斯坦所发展的广义相对论。在广义相对论中,
空间、时间和物质运动是相互作用着的。这里不但摆脱了牛顿意
义下与物质运动无关的绝对时空,也超出了狭义相对论的框架。
爱因斯坦曾说:

“空间时间未必能被看作是一种可以离开物理实在的实际
客体而独立存在的东西。物理客体不是\CJKunderdot{在空间之中},而是这些客
体有着\CJKunderdot{空间的广延}。因此,空虚的空间,这个概念就失去了它的
意义。”
