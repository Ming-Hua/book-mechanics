\achapter

\answer $ x _ { c } = 1 3 $厘米, $ y _ { c } = 8 $厘米

\answer $ x _ { c } = - 0 . 1 $米, $ y _ { c } = 0 .1 $米

\answer {\ziju{-0.05pt}取$ x $轴在二圆心联线且从圆盘中心指向孔中心的方向,取盘中心为原
  点,$ y $轴过原点与$ x $轴垂直,则得$ x _ { c } = - 5 . 6 $厘米,$ y _ { c } = 0 $}

\answer $ x _ { c } = 0 ,~ y _ { c } = - \frac { 2 a } { 8 \uppi } $

\answer (1) 气球向下运动,速率为$ \dfrac { m v } { M + m } $

(2) 气球静止

\answer $ 2.6 $米

\answer $ 4900 $公里

\answer 取两球心联线为$ x $轴,$ m _ 1 $ 球心为原点,则:

(1) $ x _ { c } \left( 0 \right) = 0 . 2 5 $米

(2) $ a _ { c } = 2 g $

(3) $ x _ { c } \left( 3 \right) = 8 8 $米

(4) $ E _ { 0 } = 8 . 9 \times 1 0 ^ 4 $焦耳

(5) $ E ^ { \prime } = 3 . 1 \times 1 0 ^ 4 $焦耳

% 392.jpg
\stepcounter{answer}
\answer (1) $ \vec{a} _ { 1 } = 3 . 5\vec{i} $米/秒\textsuperscript{2}, $ \vec{a} _ { 2 } = 2\vec{j} $米/秒\textsuperscript{2},$ \vec{a} _ { 3 } = -1.5\vec{i} $米/秒\textsuperscript{2}

(2) $ \vec{r} _ { c } \left( 0 \right) = \left( 1 . 7 5 , 0 . 2 5 \right) $

(3) $ \vec{a} = 0 . 5\vec{i} \text{米/秒\textsuperscript{2}}+1\vec{j}\text{米/秒\textsuperscript{2}}$

\answer (1) $ v _ { A } = 5 8 . 6 $米/秒, $ v _ { B } = 4 1. 4 $米/秒

(2) $ 20\% $

\answer $ f = 4 0 $ 牛顿

\answer (1) $ F / A = 1 . 8 \times 1 0 ^ 4 $牛顿/厘米\textsuperscript{2},会骨折

(2 $ F / A = 3 . 6 \times 1 0 ^ 2 $牛顿/厘米\textsuperscript{2},不会骨折

\answer $ 600 $米/秒

\answer 大于$ 0.86 $公斤$ \cdot $米/秒

\answer $ 2 . 4 \times 1 0 ^ { 5 } $米/秒\vspace{-0.5em}

\answer (1) $ \left( m _ { 1 } + m _ { 2 } \right) g + \dfrac { 2 m _ { 1 } ^ 2 } { m _ { 1 } + m _ { 2 } } g $\vspace{-0.5em}

(2) $ \Bigl( \dfrac { m _ { 1 } } { m _ { 1 } + m _ { 2 } } \Bigr) ^ { 2 } R $

\stepcounter{answer}
\answer (1) 取$ \vec{v} _ 1 $为正方向,则碰后$ \vec{v} _ { 1 } ^ { \prime } = 0 $, $ \vec{v} _ { 2 } ^ { \prime } = 1 6 $厘米/秒

(2) 取$ \vec{v} _ 1 $为正方向,碰后共同的速度$ v = 6 $厘米/秒

\answer (1) 车向右加快,$ \Delta v = 2 . 7 $米/秒

(2) 车速与他未跳下时同,仍为$ 12.7 $米/秒

\answer (1) 向右,速率为$ 4.5 $米/秒

(2)向右,速率为$ 3.5 $米/秒

\answer (1) $\displaystyle v = v _ { 0 } + \sum _ { a = 1 } ^ n \dfrac { m u } { W + a m } $

(2) $ v = v _ { 0 } + \dfrac { n m u } { W + n m } $

\answer (1) $ 8.25 \times 10 ^ 8 $米/秒

(2) $4.02\times 10 ^ 8 $米/秒

\answer 需要的力为$ \vec{F} = \left( \vec{v} - \vec{u} \right) \dfrac { \dif m } { \dif t } $

\answer 最终速率$ v = \sqrt { g \operatorname{/} k } $
% 393.jpg

\answer $\dfrac{x^{2}}{\left(\dfrac{M}{M+m}\right)^{2} R^{2}}+\dfrac{y^{2}}{R^{2}}=1 $

\answer (1) $ F = \dfrac { \dif m } { \dif t } v $,功率$ P = \dfrac { \dif m } { \dif t } v ^ { 2 } $

(2) $ \dfrac { 1 } { 2 } $转变成砂子的动能,其余由于滑动摩擦,变成热能消耗

(3) 没有改变

\addtocounter{answer}{2}
\answer $ N = \left( M + m \right) g + m g \sqrt { 1 + \dfrac { 2 k h } { g \left( M + m \right) } } $

\answer (1) $v_{A}=\sqrt{\dfrac{2 M^{2} g h \cos ^{2} \theta}{(M+m)\left(M+m \sin ^{2} \theta\right)}}$

\aindent $\vec{v}_{A}=\sqrt{\dfrac{2 m^{2} g h \cos ^{2} \theta}{(M+m)\left(M+m \sin ^{2} \theta\right)}}$

\aindent $v_{B}=\sqrt{\dfrac{2 M g h}{M+m}}$

\aindent $\vec{v}_{B}=\sqrt{\dfrac{2 m^{2} g h}{M(M+m)}}$

(2) $\Delta E_{A}=m g h\Bigl(1-\dfrac{M \cos ^{2} \theta}{M+m \sin ^{2} \theta} \Bigr)$
,这部分能量是$ m $与桌面在竖
直方向发生完全非弹性碰撞损失的

\aindent $ \Delta E _ { B } = 0 $

\answer (1) $ v _ { c0 } = \sqrt { 2 g H } $,方向沿$ AC $方向

(2) $ v _ { c } = U _ { c0 } \cos \theta $ ,方向沿水平即$ CD $方向

(3) $ \Delta P = m v _ { y } = m \sqrt { 2 g H } \sin \theta = m v _ { c0 } \sin \theta $,方向竖直向下

$ \Delta E = \dfrac { 1 } { 2 } m v _ { c0 } - \dfrac { 1 } { 2 } - m v _ { c } ^ { 2 } = m g H \sin ^ { 2 } \theta $, 这是由于在竖直方向和地
面的完全非弹性碰撞而损失的能量

\answer (1) $ s = \dfrac { h } { \mu } \left( 1 - \mu \ctg \theta \right) \left( 1 - \sin ^ 2 \theta - 2 \mu \sin \theta \cos \theta + \mu ^ { 2 } \sin ^ { 2 } \theta \right) $

(2) 当$ \theta $很小时%\vspace{-1em}
% 394.jpg
\begin{align}
   & s \approx \frac { h } { \mu } \left( 1 - \mu \ctg \theta \right) \tag{i} \\[-0.5em]
   & \ctg \theta = \frac { s _ { 0 } } { h } \tag{ii}
\end{align}

第六章习题11结果是$ s + s _ { 0 } = \dfrac { h } { \mu } $。
本习题上面的结果(i),(ii)联合,即得
$ s + s _ { 0 } \approx \dfrac { h } { \mu } - h \ctg \theta + h \ctg \theta = \dfrac { h } { \mu } $
与第六章相同

\answer (1) $ v = \sqrt { \dfrac{ 2 } { 3 } g x } $

(2) $ 2 \operatorname{/} 3 $

\answer (1) $ v = u \bigl( 1 - \e ^ { - \frac { m } { M } t } \bigr) $

(2) $ \eta = 2 \left( \dfrac { u } { v } \right) \left( 1 - \dfrac { v } { u } \right) $;$ v = \dfrac { u } { 2 } $时,$\eta$最大, $ \eta _ { \text{max}} = \dfrac { 1 } { 2 } $。式中$ \eta, v $都是数值
