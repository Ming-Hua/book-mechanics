\section{实用的惯性参考系}\label{sec:11.01}

这一章我们再回到牛顿动力学的基本问题,即动力学和参考
系的关系。

在第三章中,曾经讨论过惯性系在动力学中的特殊地位。牛
顿第一定律,实质上是肯定了自然界中存在着惯性系;牛顿第二
定律相对于惯性系才成立。因此,问题是如何找到一个惯性系。
牛顿给出了一个原则的标准。他认为存在着绝对时间和绝对空
间,那就是我们所需要的一个最基本的惯性系。然而,我们实际
上无法判定绝对时空。因此,在处理实际问题时,我们总是要找
一些具体参考系,作为实用的惯性系。在动力学中,我们不能摆
脱这些实用的惯性系来研究任何问题。所以它有非常重要的应用
价值。

如何判断一个实用的参考系是否为惯性系呢?当然,我们可
以用牛顿第一定律作为惯性系的定义来加以判断。也就是说,所
有不受外力作用的物体,都以匀速直线运动或静止的参考系为惯
性系。但问题是:如何判定一个物体不受外力呢?严格说来,自
然界中几乎找不到真正不受力的物体,只可能找到受力较小的物
体。

物体间的相互作用力有一个共同的特点,它们之间的距离越
% 318.jpg
大时,作用力越小。我们用手臂的肌肉对其他物体施力时,手臂
必须接触该物体;反之,如果手臂与物体不相接触,尽管力气再
大,该物体也不改变运动的状态。物体之间的万有引力,不要相
互接触也会有作用,但是万有引力的大小与物体间的距离平方成
反比,物体间距离越大,引力就越小。自然界中物体之间的几种
基本的相互作用,都有这种性质。

上述看法给了我们一种寻找受力较小物体的途径。只要我们
找到某物体与其周围物体相距甚远,一般说该物体受力就比较
小,选择这样的物体作为参考系,就可以认为近似是一种惯性
系,它接近于理论所定义的惯性系。这种选择实用惯性系的方法
是很有效的。下面我们介绍几种实用的惯性系。

\textsf{1. 地球}

这是我们最常用的实用惯性系,在实验室中进行的测量,一
般都是选地球参考系。因为地球与其周围物体相距比较远,距离
最近的比较重的物体是太阳,地球与太阳间距离很大,约$  1 . 5 \times 1 0 ^ { 8 }  $
公里;与月亮的距离较近,为$ 3. 8 \times 1 0 ^ { 5 }   $公里,但月亮比较小。所
以地球是一个相当好的惯性系。当然,它并不是绝对理想的惯性
系,当研究行星运动时,就会发现这一点。

\textsf{2. 太阳系}

这是一个更大的体系。以太阳系平均静止的点作为参考系的
基准,是一个更好的实用惯性系。离太阳系最近的恒星的距离约
为4光年。因为它离其他物体更远,故比地球参考系更好。当然,
它与惯性系的要求还是有偏差的,当研究恒星运动时,就会表现
出来。

\textsf{3. FK4 惯性参考系}

这是我们目前所使用的最好的实用惯性系。它选取1535颗星
体作为一个体系,把这个体系的平均不动的状态作为参考物。这
一体系离其周围其他物体更加遥远,因此更加接近理想。FK4
\clearpage\noindent
是
% 319.jpg
一个代号。最初只选了几百颗星参与平均,后来感到还不够,又
多次改进扩大星数。FK4是1960年选定的,它比前两个实用惯
性系好得多。

现在已在研究比FK4更好的惯性系。一种方案是利用一系列
射电源作为基准。射电源是目前观测到的最远的天体系统,因此,
以射电源为基准将涉及更大的范围,可能比FK4更准确。另一种
方案是利用微波背景辐射,这种辐射是均匀地猕漫在整个宇宙中
的。如果一个物体相对于背景辐射静止,那么,它将看到从不同
方向射来的背景辐射强度都相同,即所谓各向同性的,我们就可
以定义这种相对背景辐射为静止或匀速运动的体系为惯性系的基
准。这是研究宇宙问题时,最方便的一种惯性系。