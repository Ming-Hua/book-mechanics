\achapter

\answer $ 58 $公斤力,$ 5.0\times 10^2 $公斤米,或$ 4.9\times 10 ^3$焦耳

% 388.jpg
\answer $ (1) 11.76\times 1 0 ^ { 5 } $焦耳

(2) $ 0.44 $马力

\answer $ 1 . 2 3 \times 1 0 ^ { -2 }   $马力

\answer $ P = 2 3 . 2 5 $瓦,能量为$ 2.01\times 10 ^ 6 $焦耳

\answer $ 17 $秒

\answer $ 6 \times 1 0 ^ { - 2 } $米

\answer $ m g L \operatorname{/} 2 $

\answer $ 4 . 2 \times 1 0 ^ { 6 } $焦耳

\answer $ 156 $千瓦

\answer $ -22.5 $公斤$ \cdot $米

\addtocounter{answer}{2}
\answer 下落速度$v=\sqrt{g\Bigl(L-\dfrac{a^{2}}{L}\Bigr)}$

\answer $ v = 1 2 .  1 $厘米/秒


\answer $ v _ { 0 } = 7 . 1   $米/秒

\answer
$ v _ { \text{max} } = \dfrac { m _ { 2 } g } { \sqrt { m _ { 1 } k } } $

\answer $ \overline { F } = 1 2 2 0 0   $牛顿$ \approx 1.2\times 10 ^ 4 $牛顿

\answer (1) $ E _ { k 0 } = 1 8 .  6 $焦耳, $ E _ { pt } = 9 . 8   $焦耳

(2) $ f _ { \mu } = m g \sin 3 0 ^ { \circ } = 0 . 4 9   $牛顿

(3) 不会再往回滑,所以$ t = \infty $

\answer (1) $ - m g r \left( 1 - \cos \theta \right)  $

(2) $ m g r \left( 1 - \cos \theta \right) $

(3) 径向为$  2 g \left( 1 - \cos \theta \right)   $,切向为$  g \sin \theta  $

(4) $ \theta = \cos ^ { - 1 } \left( 2 / 3 \right) $

\answer (1) $ v _ { \text{min} } = \sqrt { 5 g R } $

(2) $ \theta = 1 9 ^ { \circ } 3 0 ^ { \prime }   $或 $ \theta = \sin ^ { - 1 } \left( 1 \operatorname{/} 3 \right)  $

\answer (1) $ v _ { B } = \sqrt { 2 g \left( h - R \right) }  $,$ F _ { B } = \dfrac { 2 m g \left( h - R \right) } {R} $,
$ N _ { B } = \dfrac { 2 m g h } { R } - 3 m g  $

% 389.jpg
(2) $ h \geqslant 1 . 5 R   $

\answer $ \theta = \cos ^ { - 1 } \Bigl( \dfrac { 1 } { 8 } + \sqrt { \dfrac { 1 } { 9 } - \dfrac { M } { 6 m } } \Bigr)   $

\answer (1) 绳渐伸长。在弹性限度内,升降机的动能转化为绳的弹性势能
及升降机的势能变化

(2)绳上最大张力$ T_\text{max}=9.3 $吨力伸长量$  \Delta x = 8 . 3   $厘米

\answer $ \dfrac { M } { M - m } l   $

\answer $ 11 $倍

\answer 取无穷远处为势能的零点

$ r > R,~ U \left( r \right) = - \dfrac { G M } { r }$

$ r < R,~ U \left( r \right) = - \dfrac { 3 } { 2 } \cdot \dfrac { G M } { R } + \dfrac { 1 } { 2 } \cdot \dfrac { G M } { R ^ { 2 } } r ^ { 2 } $

\stepcounter{answer}
\answer (1) $ v _ \text{逃} = \sqrt { 2 } v _ { 0 }  $

(2) 达$ R_0 $,则$  v _ { 1 } = v _ { 0 }  $;达$ \dfrac { 1 } { 2 } R _ { 0 } $,则$ v _ { 3 } = - \dfrac { 1 } { 3 } v _ { 0 }  $

(3) $v\left(R_{0}+y\right)=-m v_{0} 2\Bigl[1-\dfrac{y}{R_{0}}+\Bigl(\dfrac{y}{R_{0}}\Bigr)^{2}+\cdots\Bigr]$

(4) $v_{\text{min}}=\sqrt{2\Bigl(\dfrac{v_{0}^{2}}{R_{0}}\Bigr) y\Bigl(1-\dfrac{y}{R_{0}}\Bigr)}$,

或$v_{\text{min}} = \sqrt { 2 y g _ { 0 } } $

\stepcounter{answer}
\answer $ 9 . 6 \times 1 0 ^ { 7 } $秒

\answer (1) $ \sqrt { 2 g R _ { e } }  $,$ R_e $为地球半径

(2) $ m r + k \dot{r} - \dfrac { G M m } { r ^ { 2 } } = 0  $,取地心为坐标原点,$ r $为流星与地心之距离

(3) $ \dfrac{g m}{k} $

\answer (1) $ 5 . 7 5 \times 1 0 ^ { 16 }   $焦耳

(2) $ 5.87\times 10 ^9 $吨

% 390.jpg
\answer $ 2 . 2 3 \times 1 0 ^ { 3 }  $吨/秒
