\begin{exercises}

\exercise 宇宙射线中的$ \mu $子,以速率$  v = 0 . 9 9 c   $向地面射来。若$ \mu $子在
它为静止的参考系中,寿命$  \tau _ { 0 } = 2 . 2 2 \times 1 0 ^ { - 6 }   $秒。试问:地面上的
观察者测得$ \mu $子的平均寿命$ \tau $为多少?

\exercise 在某一惯性参考系$ K $看来,物体$ A $以匀速率$  v _ { A } = 0 . 8 c $沿$  x  $
轴正向运动,物体$ B $以匀速率$  v _ { B } = 0 . 6 c  $沿$ x $轴的负方向运动。在$ A $
看来,$ B $的速度是多少?

\exercise 一直尺,相对于$ K' $为静止,长为$ l' $,并且直尺放在与$ x' $
% 349.jpg
轴夹角为$ \theta ' $的方向上。试证明:对于$ K $中的观察者,直尺的长度
及与$ x $轴的夹角分别为
\begin{align*}
    &l=l^{\prime}\Bigg[\Bigg(\sqrt{1-\frac{v^{2}}{c^{2}}} \cos \theta^{\prime}\Bigg)^{2}+\sin ^{2} \theta^{\prime}\Bigg]^{\frac{1}{2}} \\
    &\tg \theta=\frac{\tg \theta^{\prime}}{\sqrt{1-v^{2} / c^{2}}}
\end{align*}
$ K $系与$ K' $系的定义与图\ref{fig:11.01}同。

\exercise 如果质点在$ K' $系的$ x'y' $平面中运动,其速度矢量与$ x' $的
夹角为$ \theta ' $。试证明:对于$ K $,质点的速度矢量与$ x $轴的夹角$ \theta $为
\begin{equation*}
    \tg \theta=\frac{u^{\prime} \sqrt{1-v^{2} / c^{2}} \sin \theta^{\prime}}{v+u^{\prime} \cos \theta^{\prime}}
\end{equation*}
$ K $系与$ K' $系的定义与图\ref{fig:11.01}同;上式中$ u' $为质点对$ K' $系的速度,
$ u _ x ^ { \prime } = u ^ { \prime } \cos \theta ^ { \prime } , u _ y ^ { \prime } = u ^ { \prime } \sin \theta ^ { \prime } $。

\exercise 设有一车,以匀速率$  v _ { 0 } = 1 0 0   $公里/秒作直线运动。

(1) 在车上以速率$  v _ { 1 } = 6 0   $公里/秒向前投一球,按伽利略变
换计算,站在路边的观察者看来,球的速度是多少?

(2) 在车上以速率$  v _ { 1 } = 6 0   $公里/秒向后投一球,按伽利略变
换计算,站在路边的观察者看来,球的速度是多少?

(3) 对于上述两种情况,用狭义相对论的速度合成公式,分
别求出结果。

\exercise 一高速列车以$ 0.6c $的速率沿平直轨道运行,车上有$ A $、$ B $
两人,相距$ 10.0 $米,$ A $在车的后部,$ B $在车的前部。当列车通过一
站台的时候,突然站台上的人看到$ A $先向$ B $开枪,过了$ 12.5 $毫微
秒,$ B $又向$ A $开枪。因而站台上的人作证:这场枪战是由$ A $挑起
的。假如你是车中的乘客,你看见的情况是怎样的?

\exercise 一个电子(其静止质量为$  m _ { 2 } = 9 . 1 1 \times 1 0 ^ { - 3 1 } $公斤)以$ 0.99c $
的速率运动,试问:

% 350.jpg
(1) 它的总能量是多少?

(2) 按牛顿力学算出的动能和按相对论力学算出的动能各为
多少?它们的比值是多少?

\exercise 假设一个火箭飞船的静质量为$ 8,000 $公斤,从地球飞向金
星,速率为$ 30 $公里/秒。估算一下,如果用非相对论公式计算它
的动能,则少算了多少焦耳?用这能量,能将飞船从地面升高多
少?

\exercise 一个质数为$ 42 $的静止粒子,蜕变成两个碎片,其中一
个碎片的静质量数为$ 20 $,以速率$ 3c/5 $运动。求另一碎片的动量
$ P $,能量$ E $、静质量$ m_0 $($ 1 $原子质量单位$  = 1 . 6 6 \times 1 0 ^ { - 2 7 }  $公斤)。

\exercise 在聚变过程中,四个氢核转变成一个氮核,同时以各种
辐射形式放出能量。假设一个氢核的静止质量为$ 1.0081 $原子质量
单位,而一个氦核的静止质量为$ 4.0039 $原子质量单位。计算四个
氢核聚变成一个氮核时所释放出来的能量。

\exercise 二极真空管是由一个圆筒形阳极包围一个较小的柱形
阴极构成。一个电子带着$  4 . 8 \times 1 0 ^ { - 1 6 } $焦耳的位能(相对阳极而言)
并以初速率$  v _ { 0 } = 0   $离开阴极表面。设这电子不与任何空气分子碰
撞,并且万有引力可略去不计。问:

(1)当电子撞击阳极时,它的动能有多大;

(2)电子到达阳极时的速率$ v $等于多少?

(3)如果用经典公式计算电子动能,误差有多大?

\exercise 静止的电子偶淹没时产生两个光子,如果其中一个光子
再与另一个静止电子碰撞,求它能给予这电子的最大速度。

\end{exercises}