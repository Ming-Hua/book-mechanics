\section[质能关系]{质~能~关~系}\label{sec:11.07}

本节再讨论相对论力学的另一个重要结论。我们仍从相对论
的质点运动方程\eqref{eqn:11.06.01}出发。注意其中质量是速度的函数式\lhbrak \eqref{eqn:11.06.12} \rhbrak ,所以
\begin{equation}\label{eqn:11.07.01}
    \begin{split}
        \vec{F} &= \frac {  \dif } {  \dif t } \left( m \vec{u} \right) \\
          &= m \frac {  \dif \vec{u} } {  \dif t } + \vec{u} \frac {  \dif m } {  \dif t }
    \end{split}
\end{equation}
其中$ \vec{u} $为质点速度矢量。外力$\vec{F}$作的功仍应等于质点动能的增
加,即
\begin{equation*}
    \Delta E = \int _ 1 ^ 2 \vec{F} \cdot  \dif \vec{s}
\end{equation*}
利用式\eqref{eqn:11.07.01},上式化为
\begin{equation}\label{eqn:11.07.02}
    \begin{aligned}
        \Delta E &=\int_{1}^{2} m \frac{ \dif \vec{u}}{ \dif t} \cdot  \dif \vec{s}+\int_{1}^{2} \frac{ \dif m}{ \dif t} \vec{u } \cdot  \dif \vec{s} \\
        &=\int_{1}^{2} m \vec{u} \cdot  \dif \vec{u}+\int \vec{u} \cdot \vec{u}  \dif m \\
        &=\frac{1}{2} \int_{1}^{2} m  \dif u^{2}+\int_{1}^{2} u^{2}  \dif m
    \end{aligned}
\end{equation}
其中1及2分别表示初始及终了两状态。根据式\eqref{eqn:11.06.12},可
得
\begin{equation}\label{eqn:11.07.03}
    u ^ { 2 } = c ^ { 2 } \left( 1 - \frac { m _ { 0 } ^ { 2 } } { m ^ { 2 } } \right)
\end{equation}
故
\begin{equation}\label{eqn:11.07.04}
    \frac{ \dif u^{2}}{ \dif m}=-\frac{2 m_{0}^{2} c^{2}}{m^{3}}
\end{equation}
将式\eqref{eqn:11.07.03}、\eqref{eqn:11.07.04}代入式\eqref{eqn:11.07.02},得到

% 347.jpg
~\vspace{-1.2em}
\begin{equation*}
    \begin{aligned}
        \Delta E &=\int_{1}^{2} m \frac{m_{0}^{2} c^{2}}{m^{3}}  \dif m+\int_{1}^{2} c^{2}\Bigg(1-\frac{m_{0}^{2}}{m^{2}}\Bigg)  \dif m \\
        &=\int_{1}^{2} c^{2}  \dif m
    \end{aligned}
\end{equation*}
\begin{align}\label{eqn:11.07.05}
    \beforetext{即}\Delta E = c ^ { 2 } \Delta m
\end{align}
此式表明能量的变化与质量的变化之间有简单的比例关系。它意
味着能量与质量本身之间存在着简单的比例关系,即
\begin{equation}\label{eqn:11.07.06}
    E = m c ^ { 2 }
\end{equation}
当然,如果在式\eqref{eqn:11.07.06}中附加以任何常数,仍能导致式
\eqref{eqn:11.07.05},写出式\eqref{eqn:11.07.06},即相当于取该常数为零。按照式
\eqref{eqn:11.06.12},式\eqref{eqn:11.07.06}还可写成
\vspace{-1.2em}
\begin{equation}\label{eqn:11.07.07}
    E=\frac{m_{0} c^{2}}{\sqrt{1-u^{2} / c^{2}}}
\end{equation}
如果展开成$ \dfrac { u ^ 2 } { c ^ 2 } $的幂级数,则
\begin{equation}\label{eqn:11.07.08}
    E=m_{0} c^{2}\Big\{1+\frac{1}{2} \cdot \frac{u^{2}}{c^{2}}+\frac{3}{8} \cdot \frac{u^{4}}{c^{4}}+\cdots\Big\}
\end{equation}
当$ u $小时,忽略掉所有较高级项,只保留前两项,得到
\begin{equation}\label{eqn:11.07.09}
    E \approx m _ { 0 } c ^ { 2 } + \frac { 1 } { 2 } m _ { 0 } u ^ { 2 }
\end{equation}
上式第一项是常数,第二项是质点动能项。在速度较大时,动能
已不能仅由$  \dfrac { 1 } { 2 } m _ { 0 } u ^ 2   $来表示,而应考虑$ \dfrac{3}{8} \cdot \dfrac{m_0 u^{4}}{c^{2}} $等修正项。特别
应注意的是,按式\eqref{eqn:11.07.06},当质点速度为零时,它的能量并不
为零,而是等于
\begin{equation}\label{eqn:11.07.10}
    E = v _ { o } c ^ { 2 }
\end{equation}
也就是说,即使质点没有运动,只要它的静止质量不为零,它就
已经具有能量。这个能量与静止质量成正比。这个重要的论断已
% 348.jpg
被大量的实验直接证实了。