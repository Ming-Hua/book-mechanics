\section[转动动能]{\makebox[5em][s]{转动动能}}\label{sec:09.05}

一个质点绕固定轴的平面运动,遵从方程式\eqref{eqn:09.04.02},即
\begin{equation}\label{eqn:09.05.01}
 \frac { \dif U } { \dif t } = M
\end{equation}
当质点与轴的距离$ r $固定时,当然仍有$ l = m r v $。另外,这时我
们仍可定义角速率$ \omega $为
\begin{equation*}
 \omega = \frac { v } { r }
\end{equation*}
所以,仍然有
\begin{equation*}
 l = I \omega
\end{equation*}
其中$ I = m r ^ { 2 } $是在质点距轴为$ r $时的转动惯量。方程\eqref{eqn:09.05.01}可
写成
\begin{equation}\label{eqn:09.05.02}
 \frac { \dif } { \dif t } \left( I \omega \right) = M
\end{equation}

对于多质点体系的绕轴运动,若每个质点与轴的距离可以变
化,但各质点的角速率相同,则方程\eqref{eqn:09.05.02}同样正确,转动惯
量仍由式\eqref{eqn:09.04.09}定义。由于在这种情况下$ I $不是常数,所以
方程\eqref{eqn:09.05.02}不能写成式\eqref{eqn:09.04.10}的形式。

如果外力矩$ M = 0 $ ,则由方程\eqref{eqn:09.05.02}得到
\begin{equation}\label{eqn:09.05.03}
 I \omega = \text{不变量}
\end{equation}
这个守恒律并不是新的,式\eqref{eqn:09.05.03}就是角动量守恒律在绕轴的
平面运动特殊情况下的表示。由于$ I $是可以变化的,所以我们可
以用改变$ I $的办法来改变转动速率$ \omega $。当$ I $大时$ \omega $小;当$ I $小时$ \omega $大。
% 280.jpg

用$ \omega $乘式\eqref{eqn:09.05.03}两边,得到
\begin{equation*}
 \omega \frac { \dif } { \dif t } \left( I \omega \right) = M \omega
\end{equation*}
如果$ I $不随时间变化,上式可写为
\begin{equation*}
 \frac { \dif } { \dif t } \left( \frac { 1 } { 2 } I \omega ^ { 2 } \right) = M \frac { \dif \varphi } { \dif t }
\end{equation*}
式中$ \varphi $是质点绕轴转过的角度。上式再乘以$ \dif t $,然后积分,得
\begin{equation}\label{eqn:09.05.04}
    { \left( \frac { 1 } { 2 } I \omega ^ { 2 } \right) } _ { 2 } - { \left( \frac { 1 } { 2 } I \omega ^ { 2 } \right) } _ { 1 } = \int _ 1 ^ { 2 } M \dif \varphi
\end{equation}
式\eqref{eqn:09.05.04}非常类似于动能和力的下列关系式
\begin{equation*}
    { \left( \frac { 1 } { 2 } m v ^ { 2 } \right) } _ { 2 } - { \left( \frac { 1 } { 2 } m v ^ { 2 } \right) } _ { 1 } = \int _ { 1 } ^ { 2 } F \dif x
\end{equation*}
对比二者可见,力矩与力相对应,角位移与位移相对应,转动惯
量$ I $与质量$ m $相对应,角速率$ \varphi $与速率$ v $相对应。由此,
$ \dfrac { 1 } { 2 } I \omega ^ { 2 }  $
与动能$ \dfrac { 1 } { 2 } m v ^ { 2 } $相对应,我们称前者为转动动能。表面看来,定轴
转动要比一维直线运动复杂得多,但我们看到两者所满足的方
程,以及一些有关的结论都存在着一一对应。这种现象,在物理
学中常常遇到,也就是说,在不同的物理现象之间存在着内在的
一致性。